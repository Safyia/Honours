\documentclass[a4paper,12pt]{article}

\begin{document}

  \section*{SentiWordNet}

  \begin{description}
    \item[WordNet] is a lexical database that groups English words into
    synonym sets called synsets and provides short definitions, and records
    semantic relations between the synonym sets.
          
    \item[SentiWordNet 1.0] is a lexical resource operating on top of WordNet
    synsets, in which each synset $s$ is classified based on three numerical
    scores $Neg(s)$, $Obj(s)$ and $Pos(s)$ specifying the synset's sentiment as
    either negative, objective or positive.

    \item[SentiWordNet 3.0] is newer, enhanced version of
    \textbf{SentiWordNet 1.0}
  \end{description}
     
  The method used to develop \textbf{SentiWordNet 1.0} relies on training a set
  of ternary classifiers, that is capable of deciding whether synset is
  positive, negative or objective. The ternary classifiers differs in the
  training set and the learning device used for training, therefore producing
  different classification results. Opinion-based scores for a synset are drawn
  by normalized portion of ternary classifiers that have assigned the
  corresponding label to it. If all ternary classifiers assign the same label to
  a synset, the label will have the maximum score, otherwise each label will
  have a score corresponding to number of assigned classifiers.

  Ternary classifiers are generated using semi-supervised method, where only a
  small subset $L \in Tr$ of the training data has been manually labeled. The
  training data $U = Tr - L$ were left unlabelled as it is up to the process to
  label them automatically using $L$ as an input, where
  $L = L_n \cup L_o \cup L_p$ defining $L_n$, $L_o$, and $L_p$ as sets of
  Negative, Objective and Positive synsets. $L_n$ and $L_p$ are small sets
  defined by manually selecting deliberate synsets for 14 characteristic
  positive and negative terms. $L_p$ and $L_n$ are then expanded in $K$
  interations into the final training set $Tr_{p}^{K}$ and $Tr_{n}^{K}$.

  \newpage

  The main differences between \textbf{SentiWordNet 1.0} and its newer version
  \textbf{SentiWordNet 3.0} are:
  
  \begin{itemize}
    \item Version 3.0 consists of annotation of the newer \textbf{WordNet 3.0}
    instead of the \textbf{WordNet 2.0}.

    \item In 3.0 the result of this semi-supervised learning algorithm is only
    an intermediate step of the annotation process. It is then fed to an
    iterative random-walk process that is run to convergence.

    \item The 3.0 version in both semi-supervised learning process and the
    random-walk process uses manually disambiguated glosses from
    \textbf{Princeton WordNet Gloss Corpus} which is assumed to be more accurate
    than the ones from \textbf{ExtendedWordNet} used in version 1.0.
  \end{itemize}

  The semi-supervised learning step in \textbf{SentiWordNet 3.0} is identical
  to the process used in \textbf{SentiWordNet 1.0}. In the random-walk step
  \textbf{WordNet 3.0} is viewed as a graph that is randomly, iteratively
  walked and values $Pos(s)$, $Neg(s)$ and $Obj(s)$ are expected to change
  at each iteration. The random-walk process terminates when the iteration has
  converged.
\end{document}
